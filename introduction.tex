A thermodynamically-based work potential theory for modeling progressive damage and failure in fiber-reinforced laminates is presented. The current, multiple-internal state variable (ISV) formulation, referred to as enhanced Schapery theory, utilizes separate ISVs for modeling the effects of damage and failure. Damage is considered to be the effect of any structural changes in a material that manifest as prepeak non-linearity in the stress versus strain response. Conversely, failure is taken to be the effect of the evolution of any mechanisms that results in post-peak strain softening, resulting in a negative tangent stiffness. It is assumed that matrix microdamage is the dominant damage mechanism in continuous fiber-reinforced polymer matrix laminates, and its evolution is controlled with a single ISV. Three additional ISVs are introduced to account for failure due to mode I transverse cracking, mode II transverse cracking, and mode I axial failure. Typically, failure evolution (i.e., post peak strain softening characterized through a negative tangent stiffness) results in pathologically mesh dependent solutions within a finite element (FE) framework. Therefore, consistent characteristic lengths are introduced into the formulation to govern the evolution of the three failure ISVs. Using the stationarity of the total work potential with respect to each ISV, a set of thermodynamically consistent evolution equations for the ISVs are derived. The theory is implemented in association with the commercial FE software, Abaqus. Objectivity of total energy dissipated during the failure process, with regards to refinements in the FE mesh, is demonstrated. The model is also verified against experimental results from two laminated, T800/3900-2 panels containing a central notch and different fiber orientation stacking sequences. Global load versus displacement, global load versus local strain gage data, and macroscopic failure paths obtained from the models are compared against the experimental results. \cite{pineda2009progressive,hadden2015mechanical,bednarcyk2015meso,pineda2013numerical}

A thermodynamically-based work potential theory for modeling progressive damage and failure in fiber-reinforced laminates is presented. The current, multiple-internal state variable (ISV) formulation, referred to as enhanced Schapery theory, utilizes separate ISVs for modeling the effects of damage and failure. Damage is considered to be the effect of any structural changes in a material that manifest as prepeak non-linearity in the stress versus strain response. Conversely, failure is taken to be the effect of the evolution of any mechanisms that results in post-peak strain softening, resulting in a negative tangent stiffness. It is assumed that matrix microdamage is the dominant damage mechanism in continuous fiber-reinforced polymer matrix laminates, and its evolution is controlled with a single ISV. Three additional ISVs are introduced to account for failure due to mode I transverse cracking, mode II transverse cracking, and mode I axial failure. Typically, failure evolution (i.e., post peak strain softening characterized through a negative tangent stiffness) results in pathologically mesh dependent solutions within a finite element (FE) framework. Therefore, consistent characteristic lengths are introduced into the formulation to govern the evolution of the three failure ISVs. Using the stationarity of the total work potential with respect to each ISV, a set of thermodynamically consistent evolution equations for the ISVs are derived. The theory is implemented in association with the commercial FE software, Abaqus. Objectivity of total energy dissipated during the failure process, with regards to refinements in the FE mesh, is demonstrated. The model is also verified against experimental results from two laminated, T800/3900-2 panels containing a central notch and different fiber orientation stacking sequences. Global load versus displacement, global load versus local strain gage data, and macroscopic failure paths obtained from the models are compared against the experimental results. ~\cite{hadden2015mechanical}

%%% Local Variables:
%%% mode: latex
%%% TeX-master: "main"
%%% End:
